\documentclass{beamer}
\usepackage{listings}
\usepackage{indentfirst}

\usetheme{Madrid}
\usecolortheme{default}

\title{Mixed-Integer Programming: MILP and MIQP}
\author{Muhammad Yasirroni}
\institute{Universitas Gadjah Mada}
\date{\today}

\definecolor{opt}{RGB}{203,23,206}
\definecolor{obj}{RGB}{45,177,93}
\definecolor{const}{RGB}{251,0,29}
\definecolor{var}{RGB}{18,110,213}
\definecolor{eq}{RGB}{0,0,0}

\hyphenpenalty=10000
\hbadness=10000
\tolerance=1
\emergencystretch=\maxdimen

\begin{document}

\frame{\titlepage}

\begin{frame}{Introduction}

    \textbf{Optimization}

    \begin{itemize}
        \item Optimization is a process of finding the best solution to a problem based on a set of objectives, subject to constraints.
    \end{itemize}

    \textbf{Importance of Optimization in Real-World Applications}

    \begin{itemize}
        \item Optimization is a widely used technique in various fields, such as finance, engineering, logistics, healthcare, and many others.
        \item It helps in making better decisions, improving efficiency, reducing costs, and achieving optimal outcomes.
    \end{itemize}

\end{frame}

\begin{frame}{Linear Programming}
    \begin{equation}
        \textcolor{opt}{\text{minimize~}}
        \color{obj} q^T
        \color{var} x
        \color{eq}
    \end{equation}

    \begin{equation}
        \color{const} A
        \color{var} x
        \color{const} \preceq b
        \color{eq}
    \end{equation}

    To find \textcolor{var}{decision variables} that \textcolor{opt}{minimize} (or \textcolor{opt}{maximize}) a \textcolor{obj}{linear objective function}, subject to a set of \textcolor{const}{linear constraints}.

    The $\color{var} x$ is the vector of decision variables, $\color{obj} q$ and $\color{const} b$ are vectors of coefficients, and $\color{const} A$ is a matrix of coefficients with number of rows and columns equal to the number of \textcolor{const}{linear constraints} and \textcolor{var}{decision variable}, respectively.

\end{frame}

\begin{frame}{Quadratic Programming}

    \begin{equation}
        \textcolor{opt}{\text{minimize~}}
        \color{obj} \frac{1}{2} 
        \color{var} x^T
        \color{obj} Q
        \color{var} x
        \color{eq} + 
        \color{obj} q^T
        \color{var} x
        \color{eq}
    \end{equation}

    \begin{equation}
        \color{const} A
        \color{var} x
        \color{const} \preceq b
        \color{eq}
    \end{equation}

    To find \textcolor{var}{decision variables} that \textcolor{opt}{minimize} (or \textcolor{opt}{maximize}) a \textcolor{obj}{quadratic objective function}, subject to a set of \textcolor{const}{linear constraints}.

    The $\color{var} x$ is the vector of \textcolor{var}{decision variable}, $\color{obj} q$ and $\color{const} b$ are vectors of coefficients, $\color{const} A$ is a matrix of coefficients with number of rows and columns equal to the number of \textcolor{const}{linear constraints} and \textcolor{var}{decision variable}, respectively, and $\color{obj} Q$ is a diagonal matrix of coefficients with number of rows and columns equal to the number of \textcolor{var}{decision variable}.
\end{frame}

\begin{frame}{Formulation and Components}
    \textbf{Formulation}

    \begin{itemize}
        \item The formulation involves defining \textcolor{var}{decision variables} to find, an \textcolor{obj}{objective function} to \textcolor{opt}{minimize} or \textcolor{opt}{maximize}, and \textcolor{const}{linear constraints} on the \textcolor{var}{decision variables} that must be satisfied.
        \item It needs to be noted that linear programming and quadratic programming is a problem formulation, not an algorithm. % Some example algorithms to solve linear programming are simplex, interior-point, and lagrange. Another example algorithm is branch-and-bound if there is integer decision variable.
    \end{itemize}

    \textbf{Components}

    \begin{itemize}
        \item \textcolor{var}{Decision variables} represent the quantities of the items to be produced, purchased, or allocated.
        \item \textcolor{obj}{Objective function} is an expression that defines the value to be optimized, such as profit, revenue, or cost.
        \item \textcolor{const}{Constraints} are inequalities that restrict the values of the \textcolor{var}{decision variables}, such as production capacities, material availability, and demand constraints, and must be in the form of linear expression.
    \end{itemize}
\end{frame}

\begin{frame}{Decision Variables}
    \begin{itemize}
        \item \textcolor{var}{Decision variable} can be strictly integer or continuous.
        \item If an optimization problem consist only strictly integer \textcolor{var}{decision variables}, it is called \textcolor{var}{integer} programming.
        \item If an optimization problem consist of both integer and continuous of \textcolor{var}{decision variables}, it is called \textcolor{var}{mixed-integer} programming.
    \end{itemize}
\end{frame}

\begin{frame}{Objective Functions}

    \begin{itemize}
        \item \textcolor{obj}{Quadratic} programming is \textcolor{obj}{linear} programming with \textcolor{obj}{quadratic objective function}.
        \item \textcolor{var}{mixed-integer} \textcolor{obj}{linear} programming is \textcolor{var}{mixed-integer} programming with \textcolor{obj}{linear objective function}
        \item \textcolor{var}{mixed-integer} \textcolor{obj}{quadratic} programming is \textcolor{var}{mixed-integer} programming with \textcolor{obj}{quadratic objective function}
    \end{itemize}

\end{frame}

\begin{frame}{Linear Constraints}

    \begin{itemize}
        \item Each \textcolor{const}{constraint} in \textcolor{const}{linear constraints} must be expressed in linear expressions terms. It means that each \textcolor{const}{constraint} must be expressed as a summation of \textcolor{const}{constants} that is multiplied with \textcolor{var}{decision variables} raised to the power of 1.
        \item The \textcolor{const}{constants} itself can be $0$ if a \textcolor{var}{decision variable} is not used in that \textcolor{const}{constraint}.
        \item The use of $\color{const} \preceq$ in \textcolor{const}{constraints} means that each row of $\color{const} A \color{var} x$ must be less than the value of $\color{const} b$ at that row.
    \end{itemize}

\end{frame}

\begin{frame}{Example (problem)}
    \textbf{Example: Clothing Manufacturing}

    A clothing manufacturer wants to produce summer and winter shirts, both made from cotton. The summer shirts require 2 yards of blue cotton and 3 yards of red cotton to produce. The winter shirts require 3 yards of blue cotton and 1 yard of red cotton to produce.

    The manufacturer has a total of 30 yards of blue cotton and 20 yards of red cotton available to use for production. The profit from selling one summer shirt is \$20, and the profit from selling one winter shirt is \$25.

    The manufacturer wants to maximize their profit from selling the summer and winter shirts, while using no more than the available cotton. How many summer and winter shirts should they produce to achieve this goal?
\end{frame}

\begin{frame}{Example (break downs)}

    \begin{itemize}
        \item Decision variables: number of summer and winter shirts.
        \item Objective function: maximize profit.
        \item Constraints: available cotton.
    \end{itemize}

\end{frame}

\begin{frame}{Example (decision variables)}

    To solve this problem, the manufacturer can use linear programming. Let's define the decision variables as follows:
    \begin{itemize}
        \item Let x be the number of summer shirts produced.
        \item Let y be the number of winter shirts produced.
    \end{itemize}

    The manufacturer can only produce non-negative quantities of shirts. Therefore, the first constraint is decision variables bounds:

    \begin{equation}
        0 \leq x, y \leq \infty
    \end{equation}

\end{frame}

\begin{frame}{Example (linear constraints)}

    The amount of blue cotton used for summer shirts is $2x$, and for winter shirts is $3y$. The total amount of blue cotton available is 30 yards. Therefore, the second constraint is:

    \begin{equation}
        2x + 3y \leq 30
    \end{equation}

    The amount of red cotton used for summer shirts is $3x$, and for winter shirts is $1y$. The total amount of red cotton available is 20 yards. Therefore, the third constraint is:

    \begin{equation}
        3x + y \leq 20
    \end{equation}

\end{frame}

\begin{frame}{Example (objective function)}

    The objective function that the manufacturer wants to maximize is the total profit:

    \begin{equation}
        20x + 25y
    \end{equation}

\end{frame}

\begin{frame}{Example (linear programming formulation)}

    By combining the objective function and the constraints, we can formulate the following linear programming problem:

    Maximize:

    \begin{equation}
        20x + 25y
    \end{equation}

    Subject to:

    \begin{align}
        x, y &\geq 0 \\
        x, y &\leq \infty \\
        2x + 3y &\leq 30 \\
        3x + y &\leq 20 \\
    \end{align}

\end{frame}

\begin{frame}{Acknowledgments}
    \begin{itemize}
        \item This whole presentation is made with the assistance of ChatGPT
    \end{itemize}
\end{frame}

\begin{frame}{Thank You}
    \centering
    \Huge Thank You!
\end{frame}

\end{document}
